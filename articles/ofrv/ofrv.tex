\documentclass[review]{elsarticle}

\usepackage{lineno, hyperref}
\usepackage{amssymb}
\usepackage{amsthm}
\usepackage{amsmath}
\usepackage[linesnumbered,ruled,vlined]{algorithm2e}
\theoremstyle{definition}
\newtheorem{definition}{Definition}
\theoremstyle{theorem}
\newtheorem{theorem}{Theorem}
\newtheorem{lemma}{Lemma}
\modulolinenumbers[5]

\DeclareMathOperator*{\dom}{d\omega}
\DeclareMathOperator*{\ds}{ds}
\DeclareMathOperator*{\E}{E}
\DeclareMathOperator*{\Pos}{Pos}
\DeclareMathOperator*{\Cr}{Cr}
\DeclareMathOperator*{\dr}{dr}
\DeclareMathOperator*{\dt}{dt}

\journal{Journal of \LaTeX\ Templates}

%%%%%%%%%%%%%%%%%%%%%%%
%% Elsevier bibliography styles
%%%%%%%%%%%%%%%%%%%%%%%
%% To change the style, put a % in front of the second line of the current style and
%% remove the % from the second line of the style you would like to use.
%%%%%%%%%%%%%%%%%%%%%%%

%% Numbered
%\bibliographystyle{model1-num-names}

%% Numbered without titles
%\bibliographystyle{model1a-num-names}

%% Harvard
%\bibliographystyle{model2-names.bst}\biboptions{authoryear}

%% Vancouver numbered
%\usepackage{numcompress}\bibliographystyle{model3-num-names}

%% Vancouver name/year
%\usepackage{numcompress}\bibliographystyle{model4-names}\biboptions{authoryear}

%% APA style
%\bibliographystyle{model5-names}\biboptions{authoryear}

%% AMA style
%\usepackage{numcompress}\bibliographystyle{model6-num-names}

%% `Elsevier LaTeX' style
\bibliographystyle{elsarticle-num}
%%%%%%%%%%%%%%%%%%%%%%%

\begin{document}

\begin{frontmatter}

\title{Ordered fuzzy random variables: Definition and the concept of normality}


%% Group authors per affiliation:

%% or include affiliations in footnotes:
\author[pk]{Adam Marsza\l ek\corref{mycorrespondingauthor}}
\cortext[mycorrespondingauthor]{Corresponding author}
\ead{amarszalek@pk.edu.pl}

\author[pk,ippt]{Tadeusz Burczy\'{n}ski}
\ead{tburczynski@ippt.pan.pl}

%\author[pk]{Micha\l~Bereta}
%\ead{mbereta@pk.edu.pl}

\address[pk]{Institute of Computer Science, Computational Intelligence Department, Cracow University of Technology, Cracow, Poland}
\address[ippt]{Institute of Fundamental Technological Research, Polish Academy of Sciences, Warsaw, Poland}

\begin{abstract}
The concept of fuzzy random variable combines two sources of uncertainty: randomness and fuzziness, whereas the model of ordered fuzzy numbers provides representation of inaccurate quantitative data, and is an alternative to the standard fuzzy numbers model proposed by Zadeh. This paper develops the model of ordered fuzzy numbers by defining the concept of fuzzy random variables for these numbers, called further ordered fuzzy random variables. Thanks to the well-defined arithmetic of ordered fuzzy numbers (existence of neutral and opposite elements) and the introduced ordered fuzzy random variables, it becomes possible to construct fully fuzzy stochastic time series models such as e.g. the autoregressive model or the GARCH model in the form of classical equations, which can be estimation using the least-squares or the maximum likelihood method. Furthermore, the concept of normality of ordered fuzzy random variables and the method to generate pseudo-random ordered fuzzy variables  with normal distribution are introduced.
\end{abstract}

\begin{keyword}
Ordered fuzzy numbers\sep Fuzzy random variables\sep Ordered fuzzy random variables\sep Normal ordered fuzzy random variable
\end{keyword}

\end{frontmatter}

\linenumbers

\section{Introduction}



\section{Preliminaries}
Before proceeding to formal definitions of ordered fuzzy random variables, we first review some underlying concepts: fuzzy random variables, ordered fuzzy numbers and basic concepts of stochastic processes. Readers familiar with these topics can skip this section.

\subsection{Fuzzy random variables (FRVs)}
\label{sec:frv}


\begin{definition}
\label{def:owo}
{\it Operatorem wartości oczekiwanej} zmiennej rozmytej $\tilde{a}$ nazywamy liczbę rzeczywistą $E(\tilde{a})$ taką, że
\begin{equation}
E(\tilde{a})=\int\limits_0^{\infty}\Cr\{\tilde{a}\geq r\}\dr - \int\limits_{-\infty}^0\Cr\{\tilde{a}\leq r\}\dr,
\end{equation}
pod warunkiem, że co najmniej jedna z całek jest skończona, gdzie $\Cr\{\tilde{a}\leq r\}$ jest miarą wiarygodności (ang. credibility measure) taką, że
\begin{equation}
\label{eq:cr}
\Cr\{\tilde{a}\leq r\}=1-\Cr\{\tilde{a}\geq r\} = \frac{1}{2}\left(\sup\limits_{x\leq r}\mu_{\tilde{a}}(x) + 1 - \sup\limits_{x\geq r}\mu_{\tilde{a}}(x)\right).
\end{equation}
\end{definition}   

\subsection{Ordered Fuzzy Numbers (OFNs)}
Ordered Fuzzy Numbers (called also the Kosinski's Fuzzy Numbers) introduced by Kosi\'nski et al. in series of papers \cite{kos2002,kos2003a,kos2003b,kos2004,kos2006} are defined by ordered pairs of continuous real functions defined on the interval $[0,1]$ i.e.
\begin{equation}
A=(f,g)\ \mathrm{with}\ f,g\colon[0,1]\to\mathbb{R}\ \mathrm{as\ continuous\ functions}. 
\end{equation} 

Functions $f$ and $g$ are called the \emph{up} and \emph{down}-parts of the ordered fuzzy number $A$, respectively. The continuity of both parts implies their images are bounded intervals, say \emph{UP} and \emph{DOWN}, respectively. In general, the functions $f$ and $g$ need not be invertible, and only continuity is required. If one assumes, however, that these functions are monotonous, i.e., invertible, and add the constant function of $x$ on the interval $[1_A^-,1_A^+]$ with the value equal to $1$, one might define the membership function

\begin{equation}
\label{eq:1}
\mu(x)=\left\{
\begin{array}{ccl}
f^{-1}(x) &\ \mathrm{ if }\ & x\in[f(0),f(1)],\\
g^{-1}(x) &\ \mathrm{ if }\ & x\in[g(1),g(0)],\\
1&\ \mathrm{ if }\ &x\in[1_A^-,1_A^+],
\end{array}
\right.
\end{equation}
if $f$ is increasing and $g$ is decreasing, and such that $f\leq g$ (pointwise). In this way, the obtained membership function $\mu(x),\ x\in\mathbb{R}$ represents a~mathematical object which resembles a~convex fuzzy number in the classical sense. The ordered fuzzy number and ordered fuzzy number as a~fuzzy number in classical meaning are presented in Fig.~\ref{fig:1}.

\begin{figure}[!ht]
\centering
\includegraphics[scale=0.41]{Fig1}
\vspace{-25pt}
\caption{Graphical interpretation of OFN and a OFN presented as fuzzy number in classical meaning}
\label{fig:1}
\end{figure}

In addition, note that a pair of continuous functions $(f,g)$ determines different ordered fuzzy number than the pair $(g,f)$. In this way, an extra feature to this object, named the orientation is appointed. Depending on the orientation, the ordered fuzzy numbers can be divided into two types: a~positive orientation, if the direction of ordered fuzzy number is consistent with the direction of the axis $Ox$ and a~negative orientation, if the direction of the ordered fuzzy number is opposite to the direction of the axis $Ox$.

Furthermore, the basic arithmetic operations on ordered fuzzy numbers are defined as the pairwise operations of their elements. Let $A=(f_A,g_A)$, \linebreak $B=(f_B,g_B)$ and $C=(f_C,g_C)$ are ordered fuzzy numbers. The sum \linebreak $C=A+B$, subtraction $C=A-B$, product $C= A\cdot B$, and division $C=A\div B$ are defined by formula
\begin{equation}
f_C(y)=f_A(y)\ast f_B(y),\qquad g_C(y)=g_A(y)\ast g_B(y)
\end{equation}
where $\ast$ works for $+$, $-$, $\cdot$ and $\div$, respectively, and where $C=A\div B$ is defined, if the functions $|f_B|$ and $|g_B|$ are bigger than zero. 

This definition leads to some useful properties. The one of them is existence of neutral elements of addition and multiplication. This fact causes that not always the result of an arithmetic operation is a fuzzy number with a larger support. This allows to build fuzzy models based on ordered fuzzy numbers in the form of the classical equations without losing the accuracy. In a similar way, basic math functions such as log, exp, sqrt etc. can be defined (see \cite{prokopowicz}).

Moreover, a universe $\mathcal{O}$ of all ordered fuzzy numbers can be identified with $\mathcal{C}^0([0,1])\times\mathcal{C}^0([0,1])$, hence the space $\mathcal{O}$ is topologically a Banach space \cite{kos2004}. A~class of defuzzification operators of ordered fuzzy numbers can be defined, as linear and continuous functionals on the Banach space $\mathcal{O}$. Each of them has a~representation by a sum of two Stieltjes integrals with respect to functions $\nu_1$ and $\nu_2$ of bounded variation \cite{kos2010,kos2013}.

An example of a~nonlinear functional is {\em center of gravity defuzzification} functional ($CoG$) calculated at $A=(f_A,g_A)$
\begin{equation}
\label{eq:cog}
CoG(A)=\frac{\int\limits_0^1\frac{f_A(s)+g_A(s)}{2}|f_A(s)-g_A(s)|\ds}{\int\limits_0^1|f_A(s)-g_A(s)|\ds}
\end{equation}
provided $\int\limits_0^1|f_A(s)-g_A(s)|\ds\neq0$. Center of gravity operator defined above is equivalent to the center of gravity operator in classical fuzzy logic.

In order to uniquely identify the orientation  of ordered fuzzy number, the following order operator is proposed.

\begin{definition}
Let $\mathcal{O}$ be universe of all ordered fuzzy nymbers. {\it Order operator} is defined as mapping $Ord\colon \mathcal{O}\to\{-1,0,1\}$, such that
\begin{equation}
\label{eq:os}
Ord(A)=\left\{
\begin{array}{cll}
1 & \text{if} & Defuzzy(A)>f_A(0),\\
0 & \text{if} & Defuzzy(A)=f_A(0)\ \text{or}\ Defuzzy(A)\ \text{not exist},\\
-1& \text{if} & Defuzzy(A)<f_A(0),
\end{array}
\right.
\end{equation}
where $Defuzzy(A)$ is a fixed defuzzification operator for the ordered fuzzy number $ A = (f_A, g_A) $ e.g. the center of gravity operator defined by the formula \eqref{eq:cog}. The numbers $ -1, 1, 0 $ indicate the negative, positive and lack of orientation, respectively.
\end{definition}


\subsection{Stochastic processes}
A stochastic process is defined as a collection of real-valued random  variables defined on a  common  probability  space $(\Omega, \mathcal{F}, P)$, which can be written as $\{X_t\colon t\in\mathcal{T}\}$, where $\Omega$ is  a  sample  space, $\mathcal{F}$ is a $\sigma$-algebra, and $P$ is a probability measure, and the random variables, indexed by some set $\mathcal{T}$. The set $\mathcal{T}$ is called the index set  or parameter set of the stochastic process. If the set $\mathcal{T}$ is a set of natural numbers, the process is called a discrete stochastic process. However, if $\mathcal{T}$ is a subset of the set $[0, \infty] \subset\mathbb{R}$ then process it is called a continuous stochastic process. Any stochastic process can be considered as a function of two arguments $t$ and $\omega$. In such a way that for each determined $t\in\mathcal{T}$ function $\omega\to X_t(\omega)$ is a real-valued random variable. However, for each fixed $\omega \in \Omega$ function $t\to X_t(\omega)$ is a deterministic function with real values. This function it is called the realization, trajectories or sample function of the stochastic process.\cite{florescu2014, Minier2014, GABBIANI2017}

\begin{definition}
The stochastic process $\{X_t\colon t\in\mathcal{T}\}$ is called the {\it second order stochastic process}, if for every $t\in\mathcal{T}$ the expected value of random variable $X_t^2$ exists and is finite (i.e. $\mathbb{E}[X_t^2]<\infty$).
\end{definition}

\begin{definition}
The {\it expected value} of the second order stochastic process $\{X_t\colon t\in\mathcal{T}\}$ is defined as the function $m_X\colon\mathcal{T} \to\mathbb{R}$ such that for every $t\in\mathcal{T}$
\begin{equation}
\label{eq:13}
m_X(t) = \mathbb{E}[X_t] = \int\limits_{\Omega}X_t(\omega)P(\dom).
\end{equation}
\end{definition}

\begin{definition}
The {\it variance} of the second order stochastic process \linebreak $\{X_t\colon t\in\mathcal{T}\}$ is defined as the function $\mathrm{var}_X\colon\mathcal{T}\to\mathbb{R}$ such that for every $t\in\mathcal{T}$
\begin{equation}
\mathrm{var}_X(t) = \mathrm{Var}[X_t] = \mathbb{E}[(X_t-m_X(t))^2].
\end{equation}
\end{definition}

\section{Ordered fuzzy random variables (OFRVs)}
Among the three cited definitions of fuzzy random variables in section \ref{sec:frv}, the definition proposed by Kwakernaak seems to be the most natural and the simplest in the adaptation for ordered fuzzy numbers. In this definition, for each $\alpha\in [0, 1]$, the ends of $\alpha$-cuts are identified with real-valued random variables. In case of ordered fuzzy number the ends of $\alpha$-cuts correspond to the values of functions $f(x)$ and $g(x)$, for $x = \alpha$. Hence, the ordered fuzzy random variable could be defined as the mapping from the probabilistic space into a set of all ordered fuzzy numbers, such that for each $x\in [0, 1]$, $f(x)$ and $g(x)$ are real-valued random variables. However, such a definition is very general and does not specify any relations between random variables for different values of the $x$ argument. For this reason, and because the ordered fuzzy number is an ordered pair of continuous functions, the first author in his dissertation thesis proposed definition of ordered fuzzy random variable as an ordered pair of continuous stochastic processes \cite{marszalek2017}.

Let $(\Omega, \mathcal{A}, P)$ be the specified probabilistic space and let $ mathcal{O}$ be the universe of all ordered fuzzy numbers.

\begin{definition}
{\it Ordered fuzzy random variable} is defined  as mapping  $\tilde{\xi} \colon \Omega \to \mathcal {O}$ such that $ \tilde{\xi}(\omega) = \left(f_{\tilde{\xi}}(t, \omega), g_{\tilde{\xi}}(t, \omega)\right)$ is an ordered pair of continuous stochastic processes, where $f_{\tilde{\xi}}, g_{\tilde{\xi}} \colon \mathcal{T}\times \Omega \to \mathbb{R}$, $ \mathcal{T} = [0,1] \subset \mathbb{R}$. In addition, we assume that these processes are second order stochastic processes.
\end{definition}

As can be notice, for every fixed $ t \in [0,1] $ function $ t \to \left(f_{\tilde{\xi}} (t, \omega), g_{\tilde{\xi}} (t, \omega) \right) $ is a random variable with values in $ \mathbb{R}^2 $ (random vector), and for each fixed $ \omega \in \Omega $ values of function $ \omega \to \left(f_{\tilde {\xi}} (t, \omega), g_{\tilde{\xi}} (t, \omega) \right) $ are ordered fuzzy numbers whose functions $f$ and $g$ are determined by the trajectories of stochastic processes.

For such defined ordered fuzzy random variables, terms such as expected value and variance can be defined as follows.

\begin{definition}
{\it The fuzzy expected value} of ordered fuzzy random variable $\tilde{\xi}\colon\Omega\to\mathcal{O}$ is defined as ordered fuzzy number $\mathbb{E}[\tilde{\xi}]$ such that
\begin{equation}
\mathbb{E}[\tilde{\xi}]=\left(f_{m_{\tilde{\xi}}}, g_{m_{\tilde{\xi}}} \right),
\end{equation} 
where functions $f_{m_{\tilde{\xi}}}$ and $g_{m_{\tilde{\xi}}}$ are the expected values of stochastic processes $f_{\tilde{\xi}}$ and $g_{\tilde{\xi}}$, respectively.
\end{definition}

\begin{definition}
{\it The fuzzy variance} of ordered fuzzy random variable $\tilde{\xi}\colon\Omega\to\mathcal{O}$ is defined as ordered fuzzy number $\mathrm{Var}[\tilde{\xi}]$ such that
\begin{equation}
\mathrm{Var}[\tilde{\xi}]=\left(f_{var_{\tilde{\xi}}}, g_{var_{\tilde{\xi}}} \right),
\end{equation}
where functions $f_{var_{\tilde{\xi}}}$ and $g_{var_{\tilde{\xi}}}$ are the variances of stochastic processes $f_{\tilde{\xi}}$ and $g_{\tilde{\xi}}$, respectively.
\end{definition}

It is worth notice that thanks to such definitions, basic properties of the expected value and variance are maintained.

\begin{theorem}
Suppose that exist fuzzy expected values $ \mathbb{E}[\tilde{\xi}] $ and $ \mathbb{E}[\tilde{\eta}]$ of ordered fuzzy random variables $ \tilde{\xi}$ and $ \tilde{\eta} $. Then
\begin{description}
\item[$(i)$] If $\tilde{\xi}\geq 0$, then $\mathbb{E}[\tilde{\xi}]\geq 0$.
\item[$(ii)$] $|\mathbb{E}[\tilde{\xi}]|\leq \mathbb{E}[|\tilde{\xi}|]$.
\item[$(iii)$] For $a,b\in\mathbb{R}$ fuzzy expected value of ordered fuzzy random variable $a\tilde{\xi}+b\tilde{\eta}$ exist and
$$\mathbb{E}[a\tilde{\xi}+b\tilde{\eta}]=a\mathbb{E}[\tilde{\xi}]+b\mathbb{E}[\tilde{\eta}].$$
\end{description}
\end{theorem}

\begin{theorem}
Let $\tilde{\xi}$ be ordered fuzzy random variable, for which exist fuzzy variance $\mathrm{Var}[\tilde{\xi}]$. Then
\begin{description}
\item[$(i)$] $\mathrm{Var}[\tilde{\xi}]\geq 0$.
\item[$(ii)$] For $c\in\mathbb{R}$ $\mathrm{Var}[c\tilde{\xi}]=c^2\mathrm{Var}[\tilde{\xi}]$.
\item[$(iii)$] For $c\in\mathbb{R}$ $\mathrm{Var}[\tilde{\xi} + c]=\mathrm{Var}[\tilde{\xi}]$.
\end{description} 
\end{theorem}

Proofs of the above properties follow directly from the properties of the expected value and the variance of classical real-valued random variables because for each $t\in[0,1]$ $f_{m_{\tilde{\xi}}}(t)$ and $g_{m_{\tilde{\xi}}}(t)$ are real-valued random variables. Proof of the above properties for real-valued random variables can be found in many books on the theory of probability, for example in \cite{prob2003}.

As it was noticed by Liu and Liu, in practical issues, the concept of the expected value is often required as a crisp number rather than a fuzzy number, so that it will be possible to indicate the value on the basis of which various decisions will be made. Similarly to the approach proposed by Liu and Liu, the definitions of the crisp expectation value and variance for orderd fuzzy random variables using the defuzzification operator, defined as the expected value operator, will be introduced.

\begin{definition}
{\it Expected value operator} for the ordered fuzzy number \linebreak $ A = (f_A, g_A) $ is defined as the defuzzification operator $ \E \colon \mathcal{O} \to \mathbb{R} $ specified by the formula
\begin{equation}
\E(A)=\E(f_A,g_A)=\frac{1}{2}\int\limits_0^1[f_A(s)+g_A(s)]\ds.
\end{equation}  
\end{definition}

The expected value operator defined above is equivalent to the expected value operator defined by Liu and Liu (see Def. \ref{def:owo}). This equivalence is determined by the following theorem.

\begin{theorem}
Let $A=(f,g)$ be ordered fuzzy number such that exist inverse functions $f^{-1}$, $g^{-1}$ and $f(0)\leq f(1)\leq g(1)\leq g(0)$. Which means that the ordered fuzzy number $ A $ can be represented as a classic convex fuzzy number $ A^* $ with membership function $ \mu_{A^*} \colon \mathbb{R} \to [0,1] $ as follows
\begin{equation}
\mu_{A^*}(x)=\left\{
\begin{array}{ll}
f^{-1}(x) &\text{if}\quad x\in [f(0),f(1)],\\
g^{-1}(x) &\text{if}\quad x\in [g(1),g(0)],\\
1 & \text{if}\quad x\in [f(1),g(1)],\\
0 & \text{otherwise.}
\end{array}
\right.
\end{equation}
Then the following equality is fulfilled
\begin{equation}
\label{eq:tw}
E(A^*)=\E(A),
\end{equation}
where $E$ is the operator of the expected value according to the definition \ref{def:owo}.   
\end{theorem}

\begin{proof}
According to the equation \eqref{eq:cr} the credibility measure for the fuzzy number $A^*$ is given by the formula
\begin{equation}
\label{eq:20}
\Cr\{A^*\geq t\}=\left\{
\begin{array}{cl}
1 &\text{if}\quad t < f(0),\\
1-\frac{1}{2}f^{-1}(t)&\text{if}\quad t\in [f(0),f(1)],\\
\frac{1}{2} &\text{if}\quad t\in [f(1),g(1)],\\
\frac{1}{2}g^{-1}(t) &\text{if}\quad t\in [g(1),g(0)],\\
0 &\text{if}\quad t > g(0).
\end{array}
\right.
\end{equation}
By definition, \ref{def:owo} the expected value operator of the fuzzy number $ A^* $ is given by the formula
\begin{equation}
E(A^*)=\int\limits_0^{\infty}\Cr\{A^*\geq t\}\dt - \int\limits_{-\infty}^0\Cr\{A^*\leq t\}\dt.
\end{equation}
Using the \eqref{eq:cr} formula and the integral property we get
\begin{equation}
\label{eq:22}
E(A^*)=\int\limits_{-\infty}^{\infty}\Cr\{A^*\geq t\}\dt - \int\limits_{-\infty}^01\dt.
\end{equation}
Then by inserting the formula \eqref{eq:20} into the equation \eqref{eq:22}, using the linearity of the integral, and then grouping the appropriate integrals, we get
\begin{align}
&E(A^*)= I_1 + I_2,\ \text{where} \nonumber\\
&I_1 = \int\limits_{-\infty}^{f(0)}1\dt+\int\limits_{f(0)}^{f(1)}1\dt+\frac{1}{2}\int\limits_{f(1)}^{g(1)}1\dt-\int\limits_{-\infty}^01\dt,\\
&I_2 = -\frac{1}{2}\left[\int\limits_{f(0)}^{f(1)}f^{-1}(t)\dt+\int\limits_{g(0)}^{g(1)}g^{-1}(t)\dt \right].
\end{align}  
The expression $I_1$, regardless of the location of the point $0$ relative to the points $f(0), f(1)$ and $g(1) $ is
\begin{equation}
I_1=\frac{1}{2}[f(1)+g(1)].
\end{equation}
The expression $I_2$, by substitution of appropriate integrals $s=f^{-1}(t)$ and $ s=g^{-1}(t) $, and then applying the integration by parts for both integrals is reduced to following expression
\begin{equation}
I_2 = -\frac{1}{2}[f(1)+g(1)] + \frac{1}{2}\int\limits_0^1[f(s)+g(s)]\ds.
\end{equation}  
Finally, adding up the terms $I_1$ and $I_2$ we get the right side of the equation \eqref{eq:tw}.
\end{proof}

\begin{definition}
\label{def:owo1}
The {\it crisp expected value} of the ordered fuzzy random variable $ \tilde{\xi} \colon \Omega \to \mathcal{O} $, for which existe a fuzzy expected value $ \mathbb{E}[\tilde{\xi}] $, is defined as the real number $\mathcal{E}(\tilde{\xi})$ such that
\begin{equation}
\label{eq:27}
\mathcal{E}(\tilde{\xi})=\E(\mathbb{E}[\tilde{\xi}])=\frac{1}{2}\int\limits_0^1[f_{m_{\tilde{\xi}}}(t)+g_{m_{\tilde{\xi}}}(t)]\dt.
\end{equation}
\end{definition}

The crisp expected value of the ordered fuzzy random variable can be calculated in two ways. Simply by definition, i.e. calculate the fuzzy expected value, and then defuzzy it with the expected value operator. We can also do the opposite, i.e. for fixed $\omega$ calculate the  crisp values of ordered fuzzy numbers $\tilde{\xi}(\omega)$. Then $\E(\tilde{\xi}(\omega))$ is a real-valued random variable, so we can count its expected value in the classic sense. The following theorem shows the equality of both methods.

\begin{theorem}
Let $\tilde{\xi} $ be an ordered  fuzzy random variable specified on the probabilistic space $(\Omega, \mathcal{F}, P)$. Then if exist  $\mathbb{E}[\tilde{\xi}]$, then
\begin{equation}
\label{eq:t1}
\E(\mathbb{E}[\tilde{\xi}])=\mathbb{E}[\E(\tilde{\xi})],
\end{equation}
where $\mathbb{E}$ on the right side of the equation \eqref{eq:t1} means the expected value of a classic real-valued random variable. 
\end{theorem}

\begin{proof}
According to the formula \eqref{eq:27}, using the formula \eqref{eq:13}, from the integral property and from the Fubini's theorem about iterated integrals we get
\begin{align}
\E(\mathbb{E}[\tilde{\xi}])&=\frac{1}{2}\int\limits_0^1[f_{m_{\tilde{\xi}}}(t)+g_{m_{\tilde{\xi}}}(t)]\dt=\int\limits_0^1\frac{1}{2}f_{m_{\tilde{\xi}}}(t)\dt + \int\limits_0^1\frac{1}{2}g_{m_{\tilde{\xi}}}(t)\dt\nonumber\\
&=\int\limits_0^1\mathbb{E}\left[\frac{1}{2}f_{\tilde{\xi}}(t)\right]\dt + \int\limits_0^1\mathbb{E}\left[\frac{1}{2}g_{\tilde{\xi}}(t)\right]\dt\nonumber\\
&=\int\limits_0^1\left[\int\limits_{\Omega}\frac{1}{2}f_{\tilde{\xi}}(t,\omega)P(\dom)\right]\dt + \int\limits_0^1\left[\int\limits_{\Omega}\frac{1}{2}g_{\tilde{\xi}}(t,\omega)P(\dom)\right]\dt\nonumber\\
&=\int\limits_{\Omega}\left[\frac{1}{2}\int\limits_0^1f_{\tilde{\xi}}(t,\omega)\dt+\frac{1}{2}\int\limits_0^1g_{\tilde{\xi}}(t,\omega)\dt\right]P(\dom)\nonumber\\
&=\mathbb{E}\left[\frac{1}{2}\int\limits_0^1[f_{\tilde{\xi}}(t)+g_{\tilde{\xi}}(t)]\dt\right]=\mathbb{E}[\E(\tilde{\xi})]\nonumber.
\end{align}
\end{proof}

The crisp variance of the ordered fuzzy random variable is defined as the variance of the classical real-valued random variable which is created by defuzzy ordered fuzzy random variable. Thus, it is the mean square of deviation of the defuzzy values from the crisp expected value. 

\begin{definition}
\label{def:owar}
The {\it crisp variance} of the ordered fuzzy random variable \linebreak $\tilde{\xi}\colon\Omega\to\mathcal{O}$, for which $\mathcal{E}[\tilde{\xi}]$ exists, is defined as real number $\mathcal{V}(\tilde{\xi})$ such that
\begin{equation}
\mathcal{V}(\tilde{\xi})=\mathrm{Var}[\E(\tilde{\xi})]=\mathbb{E}\left[\left(\E(\tilde{\xi})-\mathcal{E}(\tilde{\xi})\right)^2\right],
\end{equation}
where $\mathbb{E}$ and $\mathrm{Var}$ mean the expected value and the variance in the classic sense. 
\end{definition}

\section{Concept of normality of OFRVs}
From the point of view of empirical issues, the most important issue is the ability to study and generate random numbers with given probability distributions. As like the concept of a fuzzy random variable is defined by considering appropriately defined classic random variables, so defining the probability distribution of fuzzy random variable is limited to determining the distributions of these random variables \cite{puri1985,wu2009,gil2006,colubi2002,colubi2007}.

In this article, a definition of a normal probability distribution for an ordered fuzzy random variable is presented. However, unlike the classic convex fuzzy numbers, ordered fuzzy numbers have additional property, namely the orientation, which should also be taken into account.

Let $\tilde{\xi}$ be an ordered fuzzy random variable specified on probability space $(\Omega,\mathcal{F},P)$.

\begin{definition}
An ordered fuzzy random variable $\tilde{\xi}$ has {\it normal distrubution with parameters $\tilde{\mu}, \tilde{\sigma}^2\in\mathcal{O}$, $\sigma^2\in\mathbb{R}$ and $p\in [0,1]$} such that $Ord(\tilde{\mu})\geq 0$, $\tilde{\sigma}^2+\sigma^2>0$, what is denoted as $\tilde{\xi}\sim \mathcal{N}(\tilde{\mu},\tilde{\sigma}^2,\sigma^2,p)$, if for each  $t\in[0,1]$ real-valued random variables $f_{\tilde{\xi}}(t)$ and $g_{\tilde{\xi}}(t)$ have following mixed normal distrubutions
\begin{align}
&f_{\tilde{\xi}}(t)\sim p\cdot \mathcal{N}\left(f_{\tilde{\mu}}(t),f_{\tilde{\sigma}^2}(t)+\sigma^2\right)+(1-p)\cdot \mathcal{N}\left(g_{\tilde{\mu}}(t),g_{\tilde{\sigma}^2}(t)+\sigma^2\right)\nonumber\\
&g_{\tilde{\xi}}(t)\sim p\cdot \mathcal{N}\left(g_{\tilde{\mu}}(t),g_{\tilde{\sigma}^2}(t)+\sigma^2\right)+(1-p)\cdot \mathcal{N}\left(f_{\tilde{\mu}}(t),f_{\tilde{\sigma}^2}(t)+\sigma^2\right).\nonumber
\end{align}
Furthermore, $ P(Ord(\tilde{\xi}) \geq 0) = p $ and $ P(Ord(\tilde{\xi}) <0) = 1-p $, where $ Ord $ is the order operator defined using the expected value operator (see \eqref{eq:os}).
\end{definition} 

It is worth notice that unlike the classical normal distribution, the parameters $ \tilde{\mu}, \tilde{\sigma}^2 $ are not a  expected value and a variance of $ \tilde{\xi} $ (neither fuzzy nor crisp). The fuzzy expected value and fuzzy variance of $ \tilde{\xi} $ are expressed by formulas

\begin{align}
\label{eq:30}
\mathbb{E}[\tilde{\xi}]&=p\cdot(f_{\tilde{\mu}},g_{\tilde{\mu}})+(1-p)\cdot(g_{\tilde{\mu}},f_{\tilde{\mu}}).\\
\label{eq:31}
\mathrm{Var}[\tilde{\xi}]&=p(1-p)\cdot\left((f_{\tilde{\mu}},g_{\tilde{\mu}})-(g_{\tilde{\mu}},f_{\tilde{\mu}})\right)^2+p\cdot(f_{\tilde{\sigma}^2},g_{\tilde{\sigma}^2})+(1-p)\cdot (g_{\tilde{\sigma}^2},f_{\tilde{\sigma}^2})+\sigma^2.
\end{align}

In addition, estimators of parameters for the distribution defined in this way can be determined as follows. 
\begin{definition}
Let $\tilde{\xi}_1, \tilde{\xi}_2, \ldots, \tilde{\xi}_n $  be a random sample of ordered fuzzy random variables with normal distribution with parameters $ \tilde{\mu} $ , $ \tilde{\sigma}^2 $, $\sigma^2 $ and $ p $. Then
\begin{itemize}
\item Estimator of fuzzy expected value is defined as:
\begin{equation}
\hat{\mathbb{E}}[\tilde{\xi}]=\frac{1}{n}\sum\limits_{i=1}^n\tilde{\xi}_i.
\end{equation}
\item Estimator of fuzzy variance is defined as:
\begin{equation}
\label{eq:esty1}
\hat{\mathrm{Var}}[\tilde{\xi}]=\frac{1}{n-1}\sum\limits_{i=1}^n\left(\tilde{\xi}_i-\hat{\mathbb{E}}[\tilde{\xi}]\right)^2.
\end{equation}
\item Estimator of parameter $\sigma^2$ is defined as:
\begin{equation}
\label{eq:esty2}
\hat{\sigma}^2=\frac{1}{n-1}\sum\limits_{i=1}^n\left(\E(\tilde{\xi}_i)-\E(\hat{\mathbb{E}}[\tilde{\xi}])\right)^2.
\end{equation}
\item Estimator of probability $p$ is defined as:
\begin{equation}
\hat{p}=\dfrac{\#\{i\in\{1,2,\ldots,n\}\colon Ord(\tilde{\xi}_i)\geq 0\}}{n}.
\end{equation}
\item Estimators of parameter $\tilde{\mu}$ and $\tilde{\sigma}^2$ are defined as:
\begin{equation}
\hat{\tilde{\mu}}=\frac{1}{n}\sum\limits_{i=1}^n\tilde{\xi}_i^{*},
\end{equation}
\begin{equation}
\hat{\tilde{\sigma}}^2 = \frac{1}{n-1}\sum\limits_{i=1}^n\left(\tilde{\xi}_i^{*}-\hat{\tilde{\mu}}\right)^2,
\end{equation}
where
\begin{equation}
\tilde{\xi}_i^{*}=\left\{
\begin{array}{l}
(f_{\tilde{\xi}_i},g_{\tilde{\xi}_i}),\ \text{gdy}\ Ord(\tilde{\xi}_i)\geq 0,\\
(g_{\tilde{\xi}_i},f_{\tilde{\xi}_i}),\ \text{gdy}\ Ord(\tilde{\xi}_i)< 0.
\end{array}
\right.
\end{equation}
\end{itemize}
\end{definition}

In order to construct a ordered fuzzy random variable with a normal distribution with fuzzy parameters, the following lemma will be used.

\begin{lemma}
Let $\tilde{\mu}, \tilde{\sigma}^2\in\mathcal{O}$, $\sigma^2\in\mathbb{R_+}$ be fixed parameters such that $\tilde{\mu}> 0$, $\tilde{\sigma}^2>0$ (for all $t\in[0,1]$) and $Ord(\tilde{\mu})\geq 0$. Furthermore, let $\tilde{\eta}$ and $\tilde{\zeta}$ be ordered fuzzy random variables with stochastic processes $f_{\tilde{\eta}}$, $g_{\tilde{\eta}}$ and $f_{\tilde{\zeta}}$, $g_{\tilde{\zeta}}$ which are specified in following way
\begin{align}
f_{\tilde{\eta}}(t) = X_t\quad\wedge\quad g_{\tilde{\eta}}(t) = Y_t,\\
f_{\tilde{\zeta}}(t) = S\quad\wedge\quad g_{\tilde{\zeta}}(t) = S,
\end{align}
where $X_t\sim\mathcal{N}\left(1,\dfrac{f_{\tilde{\sigma}^2}(t)}{f_{\tilde{\mu}}^2(t)}\right)$, $Y_t\sim\mathcal{N}\left(1,\dfrac{g_{\tilde{\sigma}^2}(t)}{g_{\tilde{\mu}}^2(t)}\right)$ for each $t\in[0,1]$ and $S\sim\mathcal{N}\left(0,\sigma^2\right)$. 

Then, the ordered fuzzy random variable $\tilde{\xi}=\tilde{\mu}\cdot\tilde{\eta}+\tilde{\zeta}$ has normal distrubution with parameters $\tilde{\mu}_{\tilde{\xi}}=\tilde{\mu}$, $\tilde{\sigma}_{\tilde{\xi}}^2=\tilde{\sigma}^2$, $\sigma^2_{\xi}=\sigma^2$ and $p=1$.   
\end{lemma}

The proof of the above lemma implies directly from the properties of the real-valued random variables with a normal distribution such as affine property and property of sum of two normal random variables. Based on this lemma, an algorithm for generating ordered fuzzy pseudorandom numbers with a normal distribution with given parameters $\tilde{\mu}, \tilde{\sigma}^2\in\mathcal{O}$, $\sigma^2\in\mathbb{R}_+$ and $p\in[0,1]$ is presented below. For the purpose of numerical calculations, it was assumed that all ordered fuzzy numbers are determined in a discrete manner on the interval $[0, 1]$ using $T + 1$ points, $t \in \{0, 1,\ldots, T\}$.

\begin{algorithm}
  \KwIn{$\tilde{\mu}, \tilde{\sigma}^2\in\mathcal{O}$, $\sigma^2\in\mathbb{R}_+$, $p\in[0,1]$ }
  \KwOut{$\tilde{\xi}\sim \mathcal{N}(\tilde{\mu},\tilde{\sigma}^2,\sigma^2,p)$}
  \begin{description}
\item[Step 1.] Calculate constant $C:= \big|\min\{\min\limits_{t\in\{0,1,\ldots,T\}}\{f_{\tilde{\mu}}(t)\},\min\limits_{t\in\{0,1,\ldots,T\}}\{g_{\tilde{\mu}}(t)\},0\}\big|.$
\item[Step 2.] Create ordered fuzzy number $\tilde{\eta}:=\tilde{\mu}+C.$
\item[Step 3.] Generate pseudorandom real number with normal distribution $S\sim\mathcal{N}\left(0,\sigma^2\right).$
\item[Stpe 4.] Generate two sequence $X_t$ and $Y_t$ ($t\in\{0,1,\ldots,T\})$ of pseudorandom real numbers with following normal distributions
\begin{align}
&X_t\sim\mathcal{N}\left(1,\dfrac{f_{\tilde{\sigma}^2}(t)}{f_{\tilde{\eta}}^2(t)}\right),\nonumber\\
&Y_t\sim\mathcal{N}\left(1,\dfrac{g_{\tilde{\sigma}^2}(t)}{g_{\tilde{\eta}}^2(t)}\right).\nonumber
\end{align}
\item[Step 5.] Generate pseudorandom real number with uniform distribution $r\sim \mathcal{U}[0,1].$
\item[Krok 6.] \indent \\{\bf IF} $r<p$ {\bf THEN}\\ \indent  Create ordered fuzzy number $\tilde{\xi}$ such that
$$(f_{\tilde{\xi}}(t), g_{\tilde{\xi}}(t)) = (X_t, Y_t)\cdot(f_{\tilde{\eta}}(t),g_{\tilde{\eta}}(t))+S.$$
{\bf ELSE} \indent \\ \indent Create ordered fuzzy number $\tilde{\xi}$ such that
$$(f_{\tilde{\xi}}(t), g_{\tilde{\xi}}(t)) = (X_t, Y_t)\cdot(g_{\tilde{\eta}}(t),f_{\tilde{\eta}}(t))+S.$$
\item[Krok 7.] Return ordered fuzzy number $\tilde{\xi}:= \tilde{\xi} - C.$
\end{description}
\caption{}
\label{algo:1}
\end{algorithm}

\paragraph{Example} 

\section{Conclusion}

\section*{References}

\bibliography{ofrv_bibfile}

\end{document}