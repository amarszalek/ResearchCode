\documentclass[review]{elsarticle}

\usepackage{lineno, hyperref}
\usepackage{amssymb}
\usepackage{amsthm}
\usepackage{amsmath}
\theoremstyle{definition}
\newtheorem{definition}{Definition}[section]
\theoremstyle{theorem}
\newtheorem{theorem}{Theorem}[section]
\modulolinenumbers[5]

\journal{Journal of \LaTeX\ Templates}

%%%%%%%%%%%%%%%%%%%%%%%
%% Elsevier bibliography styles
%%%%%%%%%%%%%%%%%%%%%%%
%% To change the style, put a % in front of the second line of the current style and
%% remove the % from the second line of the style you would like to use.
%%%%%%%%%%%%%%%%%%%%%%%

%% Numbered
%\bibliographystyle{model1-num-names}

%% Numbered without titles
%\bibliographystyle{model1a-num-names}

%% Harvard
%\bibliographystyle{model2-names.bst}\biboptions{authoryear}

%% Vancouver numbered
%\usepackage{numcompress}\bibliographystyle{model3-num-names}

%% Vancouver name/year
%\usepackage{numcompress}\bibliographystyle{model4-names}\biboptions{authoryear}

%% APA style
%\bibliographystyle{model5-names}\biboptions{authoryear}

%% AMA style
%\usepackage{numcompress}\bibliographystyle{model6-num-names}

%% `Elsevier LaTeX' style
\bibliographystyle{elsarticle-num}
%%%%%%%%%%%%%%%%%%%%%%%

\begin{document}

\begin{frontmatter}

\title{Ordered fuzzy random variables: Definition and the concept of normality}


%% Group authors per affiliation:

%% or include affiliations in footnotes:
\author[pk]{Adam Marsza\l ek\corref{mycorrespondingauthor}}
\cortext[mycorrespondingauthor]{Corresponding author}
\ead{amarszalek@pk.edu.pl}

\author[pk,ippt]{Tadeusz Burczy\'{n}ski}
\ead{tburczynski@ippt.pan.pl}

%\author[pk]{Micha\l~Bereta}
%\ead{mbereta@pk.edu.pl}

\address[pk]{Institute of Computer Science, Computational Intelligence Department, Cracow University of Technology, Cracow, Poland}
\address[ippt]{Institute of Fundamental Technological Research, Polish Academy of Sciences, Warsaw, Poland}

\begin{abstract}
The concept of fuzzy random variable combines two sources of uncertainty: randomness and fuzziness, whereas the model of ordered fuzzy numbers provides representation of inaccurate quantitative data, and is an alternative to the standard fuzzy numbers model proposed by Zadeh. This paper develops the model of ordered fuzzy numbers by defining the concept of fuzzy random variables for these numbers, called further ordered fuzzy random variables. Thanks to the well-defined arithmetic of ordered fuzzy numbers (existence of neutral and opposite elements) and the introduced ordered fuzzy random variables, it becomes possible to construct fully fuzzy stochastic time series models such as e.g. the autoregressive model or the GARCH model in the form of classical equations, which can be estimation using the least-squares or the maximum likelihood method. Furthermore, the concept of normality of ordered fuzzy random variables and the method to generate pseudo-random ordered fuzzy variables  with normal distribution are introduced.
\end{abstract}

\begin{keyword}
Ordered fuzzy numbers\sep Fuzzy random variables\sep Ordered fuzzy random variables\sep Normal ordered fuzzy random variable
\end{keyword}

\end{frontmatter}

\linenumbers

\section{Introduction}



\section{Preliminaries}
Before proceeding to formal definitions of ordered fuzzy random variables, we first review some underlying concepts: fuzzy random variables, ordered fuzzy numbers and basic concepts of stochastic processes. Readers familiar with these topics can skip this section.

\subsection{Fuzzy random variables (FRVs)}
\label{sec:frv}

\subsection{Ordered Fuzzy Numbers (OFNs)}
Ordered Fuzzy Numbers (called also the Kosinski's Fuzzy Numbers) introduced by Kosi\'nski et al. in series of papers \cite{kos2002,kos2003a,kos2003b,kos2004,kos2006} are defined by ordered pairs of continuous real functions defined on the interval $[0,1]$ i.e.
\begin{equation}
A=(f,g)\ \mathrm{with}\ f,g\colon[0,1]\to\mathbb{R}\ \mathrm{as\ continuous\ functions}. 
\end{equation} 

Functions $f$ and $g$ are called the \emph{up} and \emph{down}-parts of the ordered fuzzy number $A$, respectively. The continuity of both parts implies their images are bounded intervals, say \emph{UP} and \emph{DOWN}, respectively. In general, the functions $f$ and $g$ need not be invertible, and only continuity is required. If one assumes, however, that these functions are monotonous, i.e., invertible, and add the constant function of $x$ on the interval $[1_A^-,1_A^+]$ with the value equal to $1$, one might define the membership function

\begin{equation}
\label{eq:1}
\mu(x)=\left\{
\begin{array}{ccl}
f^{-1}(x) &\ \mathrm{ if }\ & x\in[f(0),f(1)],\\
g^{-1}(x) &\ \mathrm{ if }\ & x\in[g(1),g(0)],\\
1&\ \mathrm{ if }\ &x\in[1_A^-,1_A^+],
\end{array}
\right.
\end{equation}
if $f$ is increasing and $g$ is decreasing, and such that $f\leq g$ (pointwise). In this way, the obtained membership function $\mu(x),\ x\in\mathbb{R}$ represents a~mathematical object which resembles a~convex fuzzy number in the classical sense. The ordered fuzzy number and ordered fuzzy number as a~fuzzy number in classical meaning are presented in Fig.~\ref{fig:1}.

\begin{figure}[!ht]
\centering
%\includegraphics[scale=0.41]{Fig1}
%%
\vspace{-25pt}
\caption{Graphical interpretation of OFN and a OFN presented as fuzzy number in classical meaning}
\label{fig:1}
\end{figure}

In addition, note that a pair of continuous functions $(f,g)$ determines different ordered fuzzy number than the pair $(g,f)$. In this way, an extra feature to this object, named the orientation is appointed. Depending on the orientation, the ordered fuzzy numbers can be divided into two types: a~positive orientation, if the direction of ordered fuzzy number is consistent with the direction of the axis $Ox$ and a~negative orientation, if the direction of the ordered fuzzy number is opposite to the direction of the axis $Ox$.

Furthermore, the basic arithmetic operations on ordered fuzzy numbers are defined as the pairwise operations of their elements. Let $A=(f_A,g_A)$, \linebreak $B=(f_B,g_B)$ and $C=(f_C,g_C)$ are ordered fuzzy numbers. The sum \linebreak $C=A+B$, subtraction $C=A-B$, product $C= A\cdot B$, and division $C=A\div B$ are defined by formula
\begin{equation}
f_C(y)=f_A(y)\ast f_B(y),\qquad g_C(y)=g_A(y)\ast g_B(y)
\end{equation}
where $\ast$ works for $+$, $-$, $\cdot$ and $\div$, respectively, and where $C=A\div B$ is defined, if the functions $|f_B|$ and $|g_B|$ are bigger than zero. 

This definition leads to some useful properties. The one of them is existence of neutral elements of addition and multiplication. This fact causes that not always the result of an arithmetic operation is a fuzzy number with a larger support. This allows to build fuzzy models based on ordered fuzzy numbers in the form of the classical equations without losing the accuracy. In a similar way, basic math functions such as log, exp, sqrt etc. can be defined (see \cite{prokopowicz}).

Moreover, a universe $\mathcal{O}$ of all ordered fuzzy numbers can be identified with $\mathcal{C}^0([0,1])\times\mathcal{C}^0([0,1])$, hence the space $\mathcal{O}$ is topologically a Banach space \cite{kos2004}. A~class of defuzzification operators of ordered fuzzy numbers can be defined, as linear and continuous functionals on the Banach space $\mathcal{O}$. Each of them has a~representation by a sum of two Stieltjes integrals with respect to functions $\nu_1$ and $\nu_2$ of bounded variation \cite{kos2010,kos2013}.

An example of a~nonlinear functional is {\em center of gravity defuzzification} functional ($CoG$) calculated at $A=(f_A,g_A)$
\begin{equation}
\label{eq:cog}
CoG(A)=\frac{\int\limits_0^1\frac{f_A(s)+g_A(s)}{2}|f_A(s)-g_A(s)|ds}{\int\limits_0^1|f_A(s)-g_A(s)|ds}
\end{equation}
provided $\int\limits_0^1|f_A(s)-g_A(s)|ds\neq0$. Center of gravity operator defined above is equivalent to the center of gravity operator in classical fuzzy logic.

\subsection{Stochastic processes}
A stochastic process is defined as a collection of real-valued random  variables defined on a  common  probability  space $(\Omega, \mathcal{F}, P)$, which can be written as $\{X_t\colon t\in\mathcal{T}\}$, where $\Omega$ is  a  sample  space, $\mathcal{F}$ is a $\sigma$-algebra, and $P$ is a probability measure, and the random variables, indexed by some set $\mathcal{T}$. The set $\mathcal{T}$ is called the index set  or parameter set of the stochastic process. If the set $\mathcal{T}$ is a set of natural numbers, the process is called a discrete stochastic process. However, if $\mathcal{T}$ is a subset of the set $[0, \infty] \subset\mathbb{R}$ then process it is called a continuous stochastic process. Any stochastic process can be considered as a function of two arguments $t$ and $\omega$. In such a way that for each determined $t\in\mathcal{T}$ function $\omega\to X_t(\omega)$ is a real-valued random variable. However, for each fixed $\omega \in \Omega$ function $t\to X_t(\omega)$ is a deterministic function with real values. This function it is called the realization, trajectories or sample function of the stochastic process.\cite{florescu2014, Minier2014, GABBIANI2017}

\begin{definition}
The stochastic process $\{X_t\colon t\in\mathcal{T}\}$ is called the {\it second order stochastic process}, if for every $t\in\mathcal{T}$ the expected value of random variable $X_t^2$ exists and is finite (i.e. $\mathbb{E}[X_t^2]<\infty$).
\end{definition}

\begin{definition}
The {\it expected value} of the second order stochastic process $\{X_t\colon t\in\mathcal{T}\}$ is defined as the function $m_X\colon\mathcal{T} \to\mathbb{R}$ such that for every $t\in\mathcal{T}$
\begin{equation}
m_X(t) = \mathbb{E}[X_t] = \int\limits_{\Omega}X_t(\omega)P(d\omega).
\end{equation}
\end{definition}

\begin{definition}
The {\it variance} of the second order stochastic process \linebreak $\{X_t\colon t\in\mathcal{T}\}$ is defined as the function $\mathrm{var}_X\colon\mathcal{T}\to\mathbb{R}$ such that for every $t\in\mathcal{T}$
\begin{equation}
\mathrm{var}_X(t) = \mathrm{Var}[X_t] = \mathbb{E}[(X_t-m_X(t))^2].
\end{equation}
\end{definition}

\section{Ordered fuzzy random variables (OFRVs)}
Among the three cited definitions of fuzzy random variables in section \ref{sec:frv}, the definition proposed by Kwakernaak seems to be the most natural and the simplest in the adaptation for ordered fuzzy numbers. In this definition, for each $\alpha\in [0, 1]$, the ends of $\alpha$-cuts are identified with real-valued random variables. In case of ordered fuzzy number the ends of $\alpha$-cuts correspond to the values of functions $f(x)$ and $g(x)$, for $x = \alpha$. Hence, the ordered fuzzy random variable could be defined as the mapping from the probabilistic space into a set of all ordered fuzzy numbers, such that for each $x\in [0, 1]$, $f(x)$ and $g(x)$ are real-valued random variables. However, such a definition is very general and does not specify any relations between random variables for different values of the $x$ argument. For this reason, and because the ordered fuzzy number is an ordered pair of continuous functions, the first author in his dissertation thesis proposed definition of ordered fuzzy random variable as an ordered pair of continuous stochastic processes \cite{marszalek2017}.

Let $(\Omega, \mathcal{A}, P)$ be the specified probabilistic space and let $ mathcal{O}$ be the universe of all ordered fuzzy numbers.

\begin{definition}
{\it Ordered fuzzy random variable} is defined  as mapping  $\tilde{\xi} \colon \Omega \to \mathcal {O}$ such that $ \tilde{\xi}(\omega) = \left(f_{\tilde{\xi}}(t, \omega), g_{\tilde{\xi}}(t, \omega)\right)$ is an ordered pair of continuous stochastic processes, where $f_{\tilde{\xi}}, g_{\tilde{\xi}} \colon \mathcal{T}\times \Omega \to \mathbb{R}$, $ \mathcal{T} = [0,1] \subset \mathbb{R}$. In addition, we assume that these processes are second order stochastic processes.
\end{definition}

As can be notice, for every fixed $ t \in [0,1] $ function $ t \to \left(f_{\tilde{\xi}} (t, \omega), g_{\tilde{\xi}} (t, \omega) \right) $ is a random variable with values in $ \mathbb{R}^2 $ (random vector), and for each fixed $ \omega \in \Omega $ values of function $ \omega \to \left(f_{\tilde {\xi}} (t, \omega), g_{\tilde{\xi}} (t, \omega) \right) $ are ordered fuzzy numbers whose functions $f$ and $g$ are determined by the trajectories of stochastic processes.

For such defined ordered fuzzy random variables, terms such as expected value and variance can be defined as follows.

\begin{definition}
{\it The fuzzy expected value} of ordered fuzzy random variable $\tilde{\xi}\colon\Omega\to\mathcal{O}$ is defined as ordered fuzzy number $\mathbb{E}[\tilde{\xi}]$ such that
\begin{equation}
\mathbb{E}[\tilde{\xi}]=\left(f_{m_{\tilde{\xi}}}, g_{m_{\tilde{\xi}}} \right),
\end{equation} 
where functions $f_{m_{\tilde{\xi}}}$ and $g_{m_{\tilde{\xi}}}$ are the expected values of stochastic processes $f_{\tilde{\xi}}$ and $g_{\tilde{\xi}}$, respectively.
\end{definition}

\begin{definition}
{\it The fuzzy variance} of ordered fuzzy random variable $\tilde{\xi}\colon\Omega\to\mathcal{O}$ is defined as ordered fuzzy number $\mathrm{Var}[\tilde{\xi}]$ such that
\begin{equation}
\mathrm{Var}[\tilde{\xi}]=\left(f_{var_{\tilde{\xi}}}, g_{var_{\tilde{\xi}}} \right),
\end{equation}
where functions $f_{var_{\tilde{\xi}}}$ and $g_{var_{\tilde{\xi}}}$ are the variances of stochastic processes $f_{\tilde{\xi}}$ and $g_{\tilde{\xi}}$, respectively.
\end{definition}

It is worth notice that thanks to such definitions, basic properties of the expected value and variance are maintained.

\begin{theorem}
Suppose that exist fuzzy expected values $ \mathbb{E}[\tilde{\xi}] $ and $ \mathbb{E}[\tilde{\eta}]$ of ordered fuzzy random variables $ \tilde{\xi}$ and $ \tilde{\eta} $. Then
\begin{description}
\item[$(i)$] If $\tilde{\xi}\geq 0$, then $\mathbb{E}[\tilde{\xi}]\geq 0$.
\item[$(ii)$] $|\mathbb{E}[\tilde{\xi}]|\leq \mathbb{E}[|\tilde{\xi}|]$.
\item[$(iii)$] For $a,b\in\mathbb{R}$ fuzzy expected value of ordered fuzzy random variable $a\tilde{\xi}+b\tilde{\eta}$ exist and
$$\mathbb{E}[a\tilde{\xi}+b\tilde{\eta}]=a\mathbb{E}[\tilde{\xi}]+b\mathbb{E}[\tilde{\eta}].$$
\end{description}
\end{theorem}

\begin{theorem}
Let $\tilde{\xi}$ be ordered fuzzy random variable, for which exist fuzzy variance $\mathrm{Var}[\tilde{\xi}]$. Then
\begin{description}
\item[$(i)$] $\mathrm{Var}[\tilde{\xi}]\geq 0$.
\item[$(ii)$] For $c\in\mathbb{R}$ $\mathrm{Var}[c\tilde{\xi}]=c^2\mathrm{Var}[\tilde{\xi}]$.
\item[$(iii)$] For $c\in\mathbb{R}$ $\mathrm{Var}[\tilde{\xi} + c]=\mathrm{Var}[\tilde{\xi}]$.
\end{description} 
\end{theorem}

Proofs of the above properties follow directly from the properties of the expected value and the variance of classical real-valued random variables because for each $t\in[0,1]$ $f_{m_{\tilde{\xi}}}(t)$ and $g_{m_{\tilde{\xi}}}(t)$ are real-valued random variables. Proof of the above properties for real-valued random variables can be found in many books on the theory of probability, for example in \cite{prob2003}.

\section{Concept of normality of OFRVs}

\section{Conclusion}

\section*{References}

\bibliography{ofrv_bibfile}

\end{document}